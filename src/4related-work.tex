\section{Related Work}

In 2009, Chen\cite{chen2009crowdsourceable} innovatively proposed the application of crowdsourcing technology to QoE (quality of experience) testing. In 2012, Liu et al.\cite{liu2012crowdsourcing} and others have combined crowdsourcing testing with usability testing. The results show that crowdsourcing usability testing can significantly reduce testing costs, but the quality of crowdsourcing availability is slightly lower than traditional usability testing. In 2013,Komarov\cite{komarov2013crowdsourcing} compared the crowdsourcing GUI test with the traditional laboratory test, and evaluated the experimental results from the results of differential distribution, task completion time, error rate and consistency, and  the feasibility of crowdsourcing GUI test. That is, crowdsourcing technology can effectively carry out GUI testing. From this point of view, most of the crowdsourcing test optimization work proposed by researchers focuses on the processing methods of test reports, and the imperfect crowdsourcing mechanism lead to heavy post-processing work. Based on this situation, this paper proposes an optimization mechanism of crowdsourcing testing process for the first time, which improves the management mode of crowdsourcing workers and optimizes the allocation of testing tasks.

