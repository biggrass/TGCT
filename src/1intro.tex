\section{Introduction}
Over the past decade, we have seen a tremendous rise in mobile applications. Given the fact that these applications are used to record and assist end users daily life, it becomes critical for developers to test Android applications and further improve software quality. However, testing on Android apps is becoming challenging due to the fragmentation issues of the Android platform\cite{park2013fragmentation}, the complexity of usage scenarios and the fast iteration of the Android platform.

Crowdsourcing was proposed to solve the problem of mobile application testing. The crowdsourced testing is that the requesters outsource the test tasks to different crowd workers, who fulfill the tasks and submit reports to the crowdsourcing test platform. The platform integrates the reports and feeds them back to the requesters\cite{feng2015test}. However, there are three inherent shortcomings of this test pattern. 1) Crowd workers have lower professional skills. 2) The test report is difficult to integrate. 3) The process of crowdsourced testing lacks management. 

Providing test guidance for crowd workers can optimize crowdsourcing testing and alleviate the problems above\cite{zhang2016guiding}. According to the idea of collective intelligence\cite{woolley2010evidence}, we developed a tool named TGCT, which helps build the GUI model and take it to guide crowd workers to verify exceptions. To verify the effectiveness of TGCT, we selected and analyzed three Android applications to complete the controlled experiment. We analyze the effectiveness of the guide mechanism and the necessity of static source code analysis. Experiment shows that the tool-guided crowdsourcing testing mechanism for mobile applications can effectively improve the quality of crowdsourcing testing.

%to do%/
%In this paper, we mainly make the following contributions:1.We firstly propose a mechanism named TGCT, which apply guide mechanism to crowdsourcing testing process. 2.Comparing the quality of the crowdsourcing test results under the guide and non-guide mechanisms. 3.Evaluate the TGCT mechanism using real data from the perspective of test efficiency and code coverage.


